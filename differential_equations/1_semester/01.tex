\documentclass[12pt]{article}
\usepackage[utf8]{inputenc}
\usepackage[T2A]{fontenc}
\usepackage[russian,english]{babel}
\usepackage{amsmath}
\usepackage{amssymb}

\title{Основные понятия и определения. Основная задача теории интегрирования и общей теории ДУ.}

\begin{document}

\maketitle

\section{Обыкновенные дифференциальные уравнения (ОДУ)}
Обыкновенное дифференциальное уравнение (ОДУ) — это уравнение вида:
\[
F(x, y, y', y'', \ldots, y^{(n)}) = 0\ (1)
\]
где:
\begin{itemize}
    \item $F$ — известная функция своих аргументов, заданная в некоторой области,
    \item $x$ — независимая переменная,
    \item $y$ — функция от $x$,
    \item $y', y'', \ldots, y^{(n)}$ — производные функции $y$.
    \item при этом $n$ действительно входит в соотношение.
\end{itemize}

\section{Уравнение с частными производными $n$-го порядка}

Пусть дана функция $u$, которая зависит от переменных $x_1, x_2, \ldots, x_n$. Тогда уравнение вида:
\[
\Phi\left(x_1, x_2, \ldots, x_n, u, \frac{\partial u}{\partial x_1}, \frac{\partial u}{\partial x_2}, \ldots, \frac{\partial u}{\partial x_n}\right) = 0\ (2)
\]
где:
\begin{itemize}
    \item $\Phi$ — известная функция своих аргументов, заданная в некоторой области,
    \item $U$ — искомая функция от переменных,
    \item $x_1, x_2, \ldots, x_n$ — независимые переменные,
    \item $\frac{\partial u}{\partial x_1}, \frac{\partial u}{\partial x_2}, \ldots, \frac{\partial u}{\partial x_n}$ — частные производные функции $u$
\end{itemize}
называется уравнением с частными производными $n$-го порядка.

\section{Решение ДУ}
\
Решением дифференциального уравнения называют такую функцию (или набор функций), 
которая при подстановке в уравнение обращает его в тождество, справедливое для всех точек упомянутой области.

В частности $y = y(x)$ будет решением ОДУ (1) в интервале $(a, b)$ если:
\[F(x, y(x), y'(x), y''(x), \ldots, y^{(n)}(x)) \equiv 0\ (3)\]
для всех $x \in (a, b)$.

\subsection{Семейство решений}
Семейством решений дифференциального уравнения называют совокупность 
всех его решений, содержащую произвольные постоянные параметры.

Уравнение первого порядка при соблюдении некоторых условий вообще 
имеет семейство решений, завсясящее от одной произвольной постоянной, 
а уравнение $n$-го порядка — от $n$ произвольных постоянных и т.д.

\subsection{Процесс нахождения решений}
Процесс нахождения решений дифференциальных уравнений называют интегрированием дифференциальных уравнений.

\section{Пример}
Пусть нам дано дифференциальное уравнение:
\[
\frac{dx}{dy} = f(t)\ (8)
\]
Пусть $f(t)$ — непрерывная функция на интервале $(a, b)$. Тогда уравнение (8) имеет общее решение вида:
\[x = \int f(t) dt + C\ (9)\]
$t \in (a, b)$, $C$ — произвольная постоянная.

Формула (9) содержит целое семейство решений, обладающих одним и тем же свойством, выраженным в уравнении (8). 
Это свойство состоит в том, что производная функции (скорость) $x$, определяемой формулой (9), равна $f(t)$ для всех $t \in (a, b)$.

Выделим из семейства решений (9) то решение, 
при котором движущая точка занимает положение $x_0$ 
в момент времени $t_0$, то есть найдем решение $x=x(t)$, удовлетворяющее условию:
\[x(t_0) = x_0 \]
\begin{itemize}
    \item $x_0$ - начальное значение искомой 
    функции (положение точки в момент времени $t_0$),
    \item $t_0$ - начальное значение аргумента (начальный момент времени),
\end{itemize}
вместе они называются начальными данными.

\section{Пример}
Материальная точка движется по горизонтальной прямой под действием силы тяжести, 
причем известны её \textbf{положение и скорость} в некоторый момент времени $t_0$.
Найти закон движения точки.

Примем нашу прямую за ось $Oy$; начало координат выберем внизу,
на уровне поверхности Земли, а направление оси $Oy$ — вверх. 
Обозначим положение точки и скорость в момент 
времени $t_0$ через $y_0$ и $v_0$ соответственно.

Принимая во внимание механический смысл второй производной, 
мы приходим к следующему дифференциальному второго порядка:
\[\frac{d^2 y}{dt^2} = -g\ (12)\]
где $g$ — ускорение свободного падения. Наша задача 
сводится к нахождению того решения $y=y(t)$ уравнения (12),
которое удовлетворяет условиям:
\[y(t_0) = y_0, \quad \frac{dy}{dt} = v_0 \quad \text{при} \quad t = t_0\ (13)\].
Числа $t_0, y_0, v_0$ — начальные данные задачи, а условия (13) — начальные условия.

Интегрируя последовательно два раза уравнение (12),
получаем общее решение:
\[\frac{dy}{dt} = -g t + C_1\ (14)\];
\[y = -\frac{g t^2}{2} + C_1 t + C_2\ (15)\].
Формула (15), где $C_1$ и $C_2$ — произвольные постоянные,
содержит все решения уравнения (12).
Выделим из этого семейства решение, удовлетворяющее начальным условиям (13).
Подставляя в (14) и (15) вместо величин $t, y, \frac{dy}{dt}$ их начальные значения 
$0, y_0, v_0$, находим: $C_1 = v_0$ и $C_2 = y_0$. Таким образом, 
\textbf{значения произвольных постоянных} $С_1$ и $С_2$  
определяются из начальных условий (13) и представляют собой 
\textbf{начальные значения искомой функции и её производной}.
Подставляя найденные значения $C_1$ и $C_2$ в (15), получаем решение задачи:
\[y = -\frac{g t^2}{2} + v_0 t + y_0\ (16)\].
Это и есть закон движения материальной точки. 

\section{Основная задача интегрирования дифференциальных уравнений}
Основная задача интегрирования дифференциальных уравнений - нахождение 
всех решений заданного дифференциального уравнения и изучение свойств этих решений.

Исключительное значение как для самой теории дифференциальных уравнений, 
так и для ее приложений имеет задача нахождения решения,
удовлетворяющего заданным условиям.

Понимание основной задачи интегрирования дифференциальных уравнений:
\begin{enumerate}
    \item Выражение искомых функций через элементареные
    \item Нахождение решений в квадратурах:
\[
y=\int \frac{\sin x}{x}dx + C
\] - уравнение в квадратурах (интегралы от элементарных функций)
    \item Поиск правила вычисления значения искомой 
    функции по заданному значению аргумента 
    (например, поиск значения в виде равномерно сходящегося ряда)
\end{enumerate}

\section{Общая задача теории дифференциальных уравнений}
Общая задача теории дифференциальных уравнений заключается в изучении
общих свойств функций, определяемых дифференциальными уравнениями,
непосредственно по виду любого заданного дифференциального уравнения,
независимо от интегрируемости этого уравнения в элементарных функциях или в квадратурах.

\end{document}