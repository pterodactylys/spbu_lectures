\documentclass[12pt]{article}
\usepackage[utf8]{inputenc}
\usepackage[T2A]{fontenc}
\usepackage[english,russian]{babel}
\usepackage{amsmath}
\usepackage{amssymb}
\usepackage[a4paper,margin=2cm]{geometry}
\usepackage{setspace}
\usepackage{hyperref}
\usepackage{mathtools}
\onehalfspacing
\sloppy
\author{Андрей}

	\title{Уравнение первого порядка, разрешённое относительно производной: решения, неявное и параметрическое задания}

\begin{document}

\maketitle

\section{Основные понятия и определения}
\subsection{Уравнение первого порядка, разрешённое относительно производной}
Уравнение первого порядка имеет вид:
\begin{equation}\label{eq:first-order-general}
    F(x,y,y') = 0,
\end{equation}
Будем рассматривать уравнение, разрешенное
относительно производной, т.е. уравнение вида:
\begin{equation}\label{eq:first-order-solvable}
    \frac{dy}{dx} = f(x,y).
\end{equation}
Так же будем рассматривать перевернутое уравнение:
\begin{equation}\label{eq:first-order-inverted}
    \frac{dx}{dy} = \frac{1}{f(x,y)},
\end{equation}
используя его в окрестностях тех точек, где
$f(x,y)$ обращается в бесконечность.

Во многих случаях целесообразно рассматривать
вместо уравнений \eqref{eq:first-order-solvable} и \eqref{eq:first-order-inverted} одно равносильное
им уравнение:
\begin{equation}\label{eq-differential-form}
    dy - f(x,y)\,dx = 0.
\end{equation}
$x,y$ в этом уравнении равноправны, то есть можно
принять за независимую переменную как $x$, так и $y$.
Умножая уравнение \eqref{eq-differential-form} на некоторую функцию $N(x,y)$,
получаем симметричное уравнение:
\begin{equation}\label{eq-symmetric}
    M(x,y)\,dx + N(x,y)\,dy = 0,
\end{equation}
где $M(x,y)=-N(x,y)f(x,y)$. Работает и обратное:
всякое уравнение вида \eqref{eq-symmetric} можно привести к виду
\eqref{eq:first-order-solvable} или \eqref{eq:first-order-inverted}, разрешая его соответственно относительно $\dfrac{dy}{dx}$
или $\dfrac{dx}{dy}$. То есть уравнение \eqref{eq-symmetric} равносильно следующим двум уравнениям:
\begin{equation}
    \frac{dy}{dx} = -\frac{M(x,y)}{N(x,y)} \quad \text{и} \quad
    \frac{dx}{dy} = -\frac{N(x,y)}{M(x,y)}.
\end{equation}
Иногда уравнение записывают в так называемой
симметрической форме:
\begin{equation}
    \frac{dx}{X(x,y)} = \frac{dy}{Y(x,y)},
\end{equation}

\subsection{Решение уравнения}
Предположим, что правая часть уравнения \eqref{eq:first-order-solvable}, 
\[\frac{dy}{dx} = f(x,y),\]
определена на некотором подмножестве
$A$ вещественной плоскости $(x,y)$.
Функцию $y=y(x)$, определенную на этом интервале,
будем называть \textbf{решением уравнения \eqref{eq:first-order-solvable} на $A$},
если:
\begin{enumerate}
    \item Существует производная $y'(x)$ для всех $x$
из интервала $(a,b)$.
    \item Функция $y=y(x)$ обращает уравнение (2)
в тождество:
\[y'(x) = f(x, y(x)),\]
справедливое для всех $x \in (a,b)$. Это значит,
что $\forall x \in (a,b)$ точка $(x,y(x))$ принадлежит
множеству $A$ и значение производной $y'(x)$
равно значению функции $f(x,y)$ в этой точке.
\end{enumerate}
Так как мы рассматриваем и перевернутое уравнение \eqref{eq:first-order-inverted},
то и решения $y=y(x)$ этого перевернутого уравнения
будем присоединять к решениям уравнения (2).

\subsection{Неявное и параметрическое задания решения}
Решение дифференциального уравнения не всегда
может быть выражено в виде явной функции $y=y(x)$.
В таких случаях решение может быть задано в \textbf{неявной форме}:
\begin{equation}\label{eq:implicit}
    \Phi(x,y)=0.
\end{equation}
Будем говорить, что это уравнение в \textbf{неявной форме}
определяет решение уравнения (2), если оно определяет
$y$ как неявную функцию от $x$, $y=y(x)$,
которая является решением уравнения (2).

В этом случае, дифференцируя уравнение \eqref{eq:implicit} по $x$ и заменяя
$dy/dx$ на $f(x,y)$, получаем:
\begin{equation}
    \frac{\partial \Phi}{\partial x} +
    \frac{\partial \Phi}{\partial y} f(x,y) = 0.
\end{equation}

Ещё одним способом задания решения
уравнения (2) является \textbf{параметрическое задание:}
\begin{equation}
    \begin{cases}
        x = \varphi(t), \\
        y = \psi(t).
    \end{cases}
\end{equation}
Будем говорить, что это задание определяет
решение уравнения (2) в параметрической форме
на интервале $(t_0, t_1)$, если на этом интервале
верно тождество:
\begin{equation}
    \frac{\psi'(t)}{\varphi'(t)} = f(\varphi(t), \psi(t)).
\end{equation}

\end{document}