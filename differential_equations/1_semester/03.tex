\documentclass[12pt]{article}
\usepackage[utf8]{inputenc}
\usepackage[T2A]{fontenc}
\usepackage[english,russian]{babel}
\usepackage{amsmath}
\usepackage{amssymb}
\usepackage[a4paper,margin=2cm]{geometry}
\usepackage{setspace}
\usepackage{hyperref}
\usepackage{mathtools}
\onehalfspacing
\sloppy

\title{Геометрическое истолкование решения уравнения}

\begin{document}

\maketitle

\section{Геометрический смысл решения дифференциального уравнения}

Будем рассматривать $x$ и $y$ как прямоугольные
координаты на плоскости. Тогда решению
$y=\varphi(x), \Phi(x, y)=0$ или $x=\varphi(t), y=\psi(t)$
уравнения
\begin{equation}\label{eq:first-order-solvable}
    \frac{dy}{dx} = f(x,y),
\end{equation}
будет соответствовать некоторая кривая на плоскости $Oxy$.
Эта кривая называется \textbf{интегральной кривой} этого уравнения.
Иногда сама кривая называется \textbf{решением уравнения} \eqref{eq:first-order-solvable}.

Будем предполагать, что интегральные кривые, о которых идет речь,
существуют.

Предположим, что правая часть уравнения \eqref{eq:first-order-solvable}
определена и конечна в каждой точке некоторой области $G$ на плоскости $Oxy$.


\end{document}